% Options for packages loaded elsewhere
\PassOptionsToPackage{unicode}{hyperref}
\PassOptionsToPackage{hyphens}{url}
%
\documentclass[
]{article}
\usepackage{amsmath,amssymb}
\usepackage{lmodern}
\usepackage{iftex}
\ifPDFTeX
  \usepackage[T1]{fontenc}
  \usepackage[utf8]{inputenc}
  \usepackage{textcomp} % provide euro and other symbols
\else % if luatex or xetex
  \usepackage{unicode-math}
  \defaultfontfeatures{Scale=MatchLowercase}
  \defaultfontfeatures[\rmfamily]{Ligatures=TeX,Scale=1}
\fi
% Use upquote if available, for straight quotes in verbatim environments
\IfFileExists{upquote.sty}{\usepackage{upquote}}{}
\IfFileExists{microtype.sty}{% use microtype if available
  \usepackage[]{microtype}
  \UseMicrotypeSet[protrusion]{basicmath} % disable protrusion for tt fonts
}{}
\makeatletter
\@ifundefined{KOMAClassName}{% if non-KOMA class
  \IfFileExists{parskip.sty}{%
    \usepackage{parskip}
  }{% else
    \setlength{\parindent}{0pt}
    \setlength{\parskip}{6pt plus 2pt minus 1pt}}
}{% if KOMA class
  \KOMAoptions{parskip=half}}
\makeatother
\usepackage{xcolor}
\IfFileExists{xurl.sty}{\usepackage{xurl}}{} % add URL line breaks if available
\IfFileExists{bookmark.sty}{\usepackage{bookmark}}{\usepackage{hyperref}}
\hypersetup{
  pdftitle={GJG Workshop - Retry Count Estimation},
  pdfauthor={Berke Kizir, Orcun Gumus},
  hidelinks,
  pdfcreator={LaTeX via pandoc}}
\urlstyle{same} % disable monospaced font for URLs
\usepackage[margin=1in]{geometry}
\usepackage{color}
\usepackage{fancyvrb}
\newcommand{\VerbBar}{|}
\newcommand{\VERB}{\Verb[commandchars=\\\{\}]}
\DefineVerbatimEnvironment{Highlighting}{Verbatim}{commandchars=\\\{\}}
% Add ',fontsize=\small' for more characters per line
\usepackage{framed}
\definecolor{shadecolor}{RGB}{248,248,248}
\newenvironment{Shaded}{\begin{snugshade}}{\end{snugshade}}
\newcommand{\AlertTok}[1]{\textcolor[rgb]{0.94,0.16,0.16}{#1}}
\newcommand{\AnnotationTok}[1]{\textcolor[rgb]{0.56,0.35,0.01}{\textbf{\textit{#1}}}}
\newcommand{\AttributeTok}[1]{\textcolor[rgb]{0.77,0.63,0.00}{#1}}
\newcommand{\BaseNTok}[1]{\textcolor[rgb]{0.00,0.00,0.81}{#1}}
\newcommand{\BuiltInTok}[1]{#1}
\newcommand{\CharTok}[1]{\textcolor[rgb]{0.31,0.60,0.02}{#1}}
\newcommand{\CommentTok}[1]{\textcolor[rgb]{0.56,0.35,0.01}{\textit{#1}}}
\newcommand{\CommentVarTok}[1]{\textcolor[rgb]{0.56,0.35,0.01}{\textbf{\textit{#1}}}}
\newcommand{\ConstantTok}[1]{\textcolor[rgb]{0.00,0.00,0.00}{#1}}
\newcommand{\ControlFlowTok}[1]{\textcolor[rgb]{0.13,0.29,0.53}{\textbf{#1}}}
\newcommand{\DataTypeTok}[1]{\textcolor[rgb]{0.13,0.29,0.53}{#1}}
\newcommand{\DecValTok}[1]{\textcolor[rgb]{0.00,0.00,0.81}{#1}}
\newcommand{\DocumentationTok}[1]{\textcolor[rgb]{0.56,0.35,0.01}{\textbf{\textit{#1}}}}
\newcommand{\ErrorTok}[1]{\textcolor[rgb]{0.64,0.00,0.00}{\textbf{#1}}}
\newcommand{\ExtensionTok}[1]{#1}
\newcommand{\FloatTok}[1]{\textcolor[rgb]{0.00,0.00,0.81}{#1}}
\newcommand{\FunctionTok}[1]{\textcolor[rgb]{0.00,0.00,0.00}{#1}}
\newcommand{\ImportTok}[1]{#1}
\newcommand{\InformationTok}[1]{\textcolor[rgb]{0.56,0.35,0.01}{\textbf{\textit{#1}}}}
\newcommand{\KeywordTok}[1]{\textcolor[rgb]{0.13,0.29,0.53}{\textbf{#1}}}
\newcommand{\NormalTok}[1]{#1}
\newcommand{\OperatorTok}[1]{\textcolor[rgb]{0.81,0.36,0.00}{\textbf{#1}}}
\newcommand{\OtherTok}[1]{\textcolor[rgb]{0.56,0.35,0.01}{#1}}
\newcommand{\PreprocessorTok}[1]{\textcolor[rgb]{0.56,0.35,0.01}{\textit{#1}}}
\newcommand{\RegionMarkerTok}[1]{#1}
\newcommand{\SpecialCharTok}[1]{\textcolor[rgb]{0.00,0.00,0.00}{#1}}
\newcommand{\SpecialStringTok}[1]{\textcolor[rgb]{0.31,0.60,0.02}{#1}}
\newcommand{\StringTok}[1]{\textcolor[rgb]{0.31,0.60,0.02}{#1}}
\newcommand{\VariableTok}[1]{\textcolor[rgb]{0.00,0.00,0.00}{#1}}
\newcommand{\VerbatimStringTok}[1]{\textcolor[rgb]{0.31,0.60,0.02}{#1}}
\newcommand{\WarningTok}[1]{\textcolor[rgb]{0.56,0.35,0.01}{\textbf{\textit{#1}}}}
\usepackage{graphicx}
\makeatletter
\def\maxwidth{\ifdim\Gin@nat@width>\linewidth\linewidth\else\Gin@nat@width\fi}
\def\maxheight{\ifdim\Gin@nat@height>\textheight\textheight\else\Gin@nat@height\fi}
\makeatother
% Scale images if necessary, so that they will not overflow the page
% margins by default, and it is still possible to overwrite the defaults
% using explicit options in \includegraphics[width, height, ...]{}
\setkeys{Gin}{width=\maxwidth,height=\maxheight,keepaspectratio}
% Set default figure placement to htbp
\makeatletter
\def\fps@figure{htbp}
\makeatother
\setlength{\emergencystretch}{3em} % prevent overfull lines
\providecommand{\tightlist}{%
  \setlength{\itemsep}{0pt}\setlength{\parskip}{0pt}}
\setcounter{secnumdepth}{-\maxdimen} % remove section numbering
\ifLuaTeX
  \usepackage{selnolig}  % disable illegal ligatures
\fi

\title{GJG Workshop - Retry Count Estimation}
\author{Berke Kizir, Orcun Gumus}
\date{November 9, 2022}

\begin{document}
\maketitle

\hypertarget{introduction}{%
\subsection{1. Introduction}\label{introduction}}

Zen Match is our causal game introduced at 2021. Each day nearly 2
million user plays the game generate TBs of data. Today we prepared a
small dataset from the data and we prepared a challenge to solve.

As we discussed in our blog difficulty effects the gaming experience. In
short, skilled players prefer the game harder while new comers in the
genre would rather have easier experience, see
\url{https://science.goodjobgmaes.com} for more.

\begin{figure}
\centering
\includegraphics{https://gjg-data-science-public.s3.amazonaws.com/ds-ws/skills-difficulty.gif}
\caption{Figure 1: The skills of the segments - we assume constant
across the levels, which is apparently not in the case in real
historical data -, and the skill difficulty difference of the different
segments. That shows newbies enters the frustration much faster than
average players, while skilled players are always in boredom area.}
\end{figure}

The aim of this workshop to understand the players abilities, skills,
using the real historical data. We will try to estimate how many retry
the players will be in need to pass the levels before they reach that
specific level.

At the end of the workshop we prepare a betting game. The betting sheet
contains pid(players id), lid(level), ou(over/under), (retry)retry
quoted value, odd. The question is would you prefer to play the bet or
not. At the end we will compare the profit and loss of our findings
compared to betting randomly.

\hypertarget{libraries-that-we-need}{%
\subsection{2. Libraries that we need}\label{libraries-that-we-need}}

\begin{Shaded}
\begin{Highlighting}[]
\FunctionTok{library}\NormalTok{(dsws)}
\end{Highlighting}
\end{Shaded}

\begin{Shaded}
\begin{Highlighting}[]
\CommentTok{\#install.packages("vroom")}
\CommentTok{\#install.packages("foreach")}
\CommentTok{\#install.packages("tidyverse")}

\FunctionTok{library}\NormalTok{(}\StringTok{"vroom"}\NormalTok{)}
\FunctionTok{library}\NormalTok{(}\StringTok{"tidyverse"}\NormalTok{)}
\end{Highlighting}
\end{Shaded}

\begin{verbatim}
## Registered S3 methods overwritten by 'readr':
##   method                    from 
##   as.data.frame.spec_tbl_df vroom
##   as_tibble.spec_tbl_df     vroom
##   format.col_spec           vroom
##   print.col_spec            vroom
##   print.collector           vroom
##   print.date_names          vroom
##   print.locale              vroom
##   str.col_spec              vroom
\end{verbatim}

\begin{verbatim}
## -- Attaching packages --------------------------------------- tidyverse 1.3.2 --
## v ggplot2 3.3.6     v purrr   0.3.4
## v tibble  3.1.7     v dplyr   1.0.9
## v tidyr   1.2.0     v stringr 1.4.0
## v readr   2.1.2     v forcats 0.5.1
## -- Conflicts ------------------------------------------ tidyverse_conflicts() --
## x readr::col_character()   masks vroom::col_character()
## x readr::col_date()        masks vroom::col_date()
## x readr::col_datetime()    masks vroom::col_datetime()
## x readr::col_double()      masks vroom::col_double()
## x readr::col_factor()      masks vroom::col_factor()
## x readr::col_guess()       masks vroom::col_guess()
## x readr::col_integer()     masks vroom::col_integer()
## x readr::col_logical()     masks vroom::col_logical()
## x readr::col_number()      masks vroom::col_number()
## x readr::col_skip()        masks vroom::col_skip()
## x readr::col_time()        masks vroom::col_time()
## x readr::cols()            masks vroom::cols()
## x readr::date_names_lang() masks vroom::date_names_lang()
## x readr::default_locale()  masks vroom::default_locale()
## x dplyr::filter()          masks stats::filter()
## x readr::fwf_cols()        masks vroom::fwf_cols()
## x readr::fwf_empty()       masks vroom::fwf_empty()
## x readr::fwf_positions()   masks vroom::fwf_positions()
## x readr::fwf_widths()      masks vroom::fwf_widths()
## x dplyr::lag()             masks stats::lag()
## x readr::locale()          masks vroom::locale()
## x readr::output_column()   masks vroom::output_column()
## x readr::problems()        masks vroom::problems()
\end{verbatim}

\begin{Shaded}
\begin{Highlighting}[]
\FunctionTok{library}\NormalTok{(}\StringTok{"foreach"}\NormalTok{)}
\end{Highlighting}
\end{Shaded}

\begin{verbatim}
## 
## Attaching package: 'foreach'
## 
## The following objects are masked from 'package:purrr':
## 
##     accumulate, when
\end{verbatim}

The tidyverse is an opinionated collection of R packages designed for
data science, We will use dplyr and ggplot packages. For foreach
functionality we need foreach package,

\begin{Shaded}
\begin{Highlighting}[]
\CommentTok{\#install.packages("cmdstanr", repos = c("https://mc{-}stan.org/r{-}packages/", getOption("repos")))}
\CommentTok{\#install\_cmdstan(dir = "\textasciitilde{}/.cmdstan/cmdstan{-}2.30.1", cores = getOption("mc.cores", 6), overwrite = TRUE, version = "2.30.1", quiet = TRUE)}

\FunctionTok{library}\NormalTok{(cmdstanr)}
\end{Highlighting}
\end{Shaded}

\begin{verbatim}
## This is cmdstanr version 0.5.3
\end{verbatim}

\begin{verbatim}
## - CmdStanR documentation and vignettes: mc-stan.org/cmdstanr
\end{verbatim}

\begin{verbatim}
## - CmdStan path: /Users/guemues/.cmdstanr/cmdstan-2.29.2
\end{verbatim}

\begin{verbatim}
## - CmdStan version: 2.29.2
\end{verbatim}

\begin{verbatim}
## 
## A newer version of CmdStan is available. See ?install_cmdstan() to install it.
## To disable this check set option or environment variable CMDSTANR_NO_VER_CHECK=TRUE.
\end{verbatim}

We need cmdstanr and cmdstan in order to do bayesian inference,

\hypertarget{data-setup}{%
\subsection{3. Data setup}\label{data-setup}}

In the dataset we have level id, which also sorted and shows progress.
Retry count and superundo and shuffle usage in the level. Superundo and
shuffle help the players and increase the probabilirty of passing the
level.

\hypertarget{betting-randomly}{%
\subsubsection{3.1 Betting randomly}\label{betting-randomly}}

Lets see what playing randomly resulted

\begin{Shaded}
\begin{Highlighting}[]
\FunctionTok{set.seed}\NormalTok{(}\DecValTok{4}\NormalTok{)}
\NormalTok{random\_play\_vector }\OtherTok{\textless{}{-}} \FunctionTok{sample}\NormalTok{(}\FunctionTok{c}\NormalTok{(}\DecValTok{0}\NormalTok{,}\DecValTok{1}\NormalTok{), }\AttributeTok{replace=}\ConstantTok{TRUE}\NormalTok{, }\AttributeTok{size=}\FunctionTok{nrow}\NormalTok{(odds\_table))}
\NormalTok{dsws}\SpecialCharTok{::}\FunctionTok{score\_the\_play\_vector}\NormalTok{(random\_play\_vector, }\AttributeTok{scores =}\NormalTok{ dsws}\SpecialCharTok{::}\NormalTok{scores, }\AttributeTok{odds\_table =}\NormalTok{ dsws}\SpecialCharTok{::}\NormalTok{odds\_table)}
\end{Highlighting}
\end{Shaded}

\includegraphics{workshop_notebook_files/figure-latex/unnamed-chunk-5-1.pdf}

\begin{verbatim}
## TableGrob (1 x 2) "arrange": 2 grobs
##   z     cells    name           grob
## 1 1 (1-1,1-1) arrange gtable[layout]
## 2 2 (1-1,2-2) arrange gtable[layout]
\end{verbatim}

\hypertarget{short-eda-on-train-test-data}{%
\subsubsection{3.2 Short EDA on train test
data}\label{short-eda-on-train-test-data}}

There is 150 users in the dataset and 100 levels ranging from 100 to
200. 50 users and 50 levels used in testing. For users 100 to 150 we do
not have the retry information from 150 to 200.

\begin{Shaded}
\begin{Highlighting}[]
\FunctionTok{ggplot}\NormalTok{(train\_data }\SpecialCharTok{\%\textgreater{}\%} \FunctionTok{mutate}\NormalTok{(}\AttributeTok{retrial\_count=} \FunctionTok{pmin}\NormalTok{(retrial\_count, }\DecValTok{10}\NormalTok{)), }\FunctionTok{aes}\NormalTok{(}\AttributeTok{x=}\NormalTok{level\_id, }\AttributeTok{y=}\NormalTok{ user\_id, }\AttributeTok{fill=}\NormalTok{retrial\_count)) }\SpecialCharTok{+} 
  \FunctionTok{geom\_tile}\NormalTok{()}
\end{Highlighting}
\end{Shaded}

\includegraphics{workshop_notebook_files/figure-latex/unnamed-chunk-6-1.pdf}

Here we see that blue dots are getting more dense while levels are
increasing, and some players are passign the levels without retrying.

Lets check our assumptions with aggregated data, is the game getting
harder with new levels:

\begin{Shaded}
\begin{Highlighting}[]
\NormalTok{retry\_counts\_per\_level }\OtherTok{\textless{}{-}}\NormalTok{ train\_data }\SpecialCharTok{\%\textgreater{}\%} \FunctionTok{group\_by}\NormalTok{(level\_id) }\SpecialCharTok{\%\textgreater{}\%} \FunctionTok{summarise}\NormalTok{(}\AttributeTok{retrial\_count=}\FunctionTok{sum}\NormalTok{(retrial\_count))}
\FunctionTok{ggplot}\NormalTok{(}\AttributeTok{data=}\NormalTok{retry\_counts\_per\_level, }\FunctionTok{aes}\NormalTok{(}\AttributeTok{x=}\NormalTok{level\_id, }\AttributeTok{y=}\NormalTok{ retrial\_count)) }\SpecialCharTok{+} 
  \FunctionTok{geom\_line}\NormalTok{() }\SpecialCharTok{+}
  \FunctionTok{geom\_smooth}\NormalTok{(}\AttributeTok{span =} \FloatTok{0.2}\NormalTok{)}\SpecialCharTok{+}
  \FunctionTok{geom\_smooth}\NormalTok{(}\AttributeTok{method =}\NormalTok{ lm, }\AttributeTok{se =} \ConstantTok{FALSE}\NormalTok{)}
\end{Highlighting}
\end{Shaded}

\begin{verbatim}
## `geom_smooth()` using method = 'loess' and formula 'y ~ x'
\end{verbatim}

\begin{verbatim}
## `geom_smooth()` using formula 'y ~ x'
\end{verbatim}

\includegraphics{workshop_notebook_files/figure-latex/unnamed-chunk-7-1.pdf}

\hypertarget{modelling-the-data}{%
\subsection{4. Modelling the data}\label{modelling-the-data}}

\hypertarget{modelling-only-with-level-and-user}{%
\subsubsection{4.1 Modelling only with level and
user}\label{modelling-only-with-level-and-user}}

Our assumption is to retry count in level i by user j, \(t_{ij}\) is
distributed with a geometric distribution
\[  t_{ij} \sim Geometric (p_{ij}) \] where p is the probability of the
successful passing in one retry of level j by user i. If we manage to
estimate \(p_{ij}\) we can estimate how many retries will be required by
the user i for level j to pass.

To start with the simplest case let's assume \(p_{ij}\) dependent only
on the level and the user. In other words, each player has different
equal distribution on the level j.

\[  p_{ij} = inv\_logit(intercept + u_i + l_j) \]

\[  u_i \sim normal (0, alpha_u) \] \[  l_j \sim normal (0, alpha_l) \]
\[  alpha_u \sim exponential (1) \] \[  alpha_l \sim exponential (1) \]

\(alpha_u\) and \(alpha_l\) are hyper priors for multi level model. We
only observe \(t_{ij}\)while other parameters have latent effects. Check
\url{https://nicholasrjenkins.science/tutorials/bayesian-inference-with-stan/mm_stan/}
for more on stan and multi level modelling.

Lets first define the constant across trainings and simulations. You can
change chains and iter sampling acording to you computer performance.

\begin{Shaded}
\begin{Highlighting}[]
\NormalTok{ITER\_SAMPLING }\OtherTok{=} \DecValTok{150}
\NormalTok{CHAINS }\OtherTok{=} \DecValTok{4}
\NormalTok{MIN\_TEST\_USER\_ID }\OtherTok{=} \FunctionTok{min}\NormalTok{(dsws}\SpecialCharTok{::}\NormalTok{test\_data}\SpecialCharTok{$}\NormalTok{user\_id)}
\NormalTok{MIN\_TEST\_LEVEL\_ID }\OtherTok{=} \FunctionTok{min}\NormalTok{(dsws}\SpecialCharTok{::}\NormalTok{test\_data}\SpecialCharTok{$}\NormalTok{level\_id)}
\NormalTok{SAMPLE\_COUNT }\OtherTok{=}\NormalTok{ ITER\_SAMPLING }\SpecialCharTok{*}\NormalTok{ CHAINS}
\NormalTok{FIRST\_LEVEL }\OtherTok{=} \FunctionTok{min}\NormalTok{(dsws}\SpecialCharTok{::}\NormalTok{train\_data}\SpecialCharTok{$}\NormalTok{level\_id)}
\end{Highlighting}
\end{Shaded}

\begin{Shaded}
\begin{Highlighting}[]
\NormalTok{model\_1 }\OtherTok{\textless{}{-}} \FunctionTok{cmdstan\_model}\NormalTok{(}\StringTok{\textquotesingle{}./geometric\_model\_00.stan\textquotesingle{}}\NormalTok{)}

\NormalTok{input\_list }\OtherTok{\textless{}{-}} \FunctionTok{list}\NormalTok{(}
  \AttributeTok{N =} \FunctionTok{nrow}\NormalTok{(dsws}\SpecialCharTok{::}\NormalTok{train\_data),}
  \AttributeTok{user\_id =}\NormalTok{ dsws}\SpecialCharTok{::}\NormalTok{train\_data}\SpecialCharTok{$}\NormalTok{user\_id,}
  \AttributeTok{N\_of\_user\_id =} \FunctionTok{max}\NormalTok{(dsws}\SpecialCharTok{::}\NormalTok{train\_data}\SpecialCharTok{$}\NormalTok{user\_id),}
  
  \AttributeTok{level\_id =}\NormalTok{ dsws}\SpecialCharTok{::}\NormalTok{train\_data}\SpecialCharTok{$}\NormalTok{level\_id }\SpecialCharTok{{-}}\NormalTok{ FIRST\_LEVEL }\SpecialCharTok{+} \DecValTok{1}\NormalTok{,}
  \AttributeTok{N\_of\_level\_id =} \FunctionTok{max}\NormalTok{(dsws}\SpecialCharTok{::}\NormalTok{train\_data}\SpecialCharTok{$}\NormalTok{level\_id) }\SpecialCharTok{{-}}\NormalTok{ FIRST\_LEVEL }\SpecialCharTok{+} \DecValTok{1}\NormalTok{,}
  
  \AttributeTok{retrial\_count =}\NormalTok{ dsws}\SpecialCharTok{::}\NormalTok{train\_data}\SpecialCharTok{$}\NormalTok{retrial\_count,}
  
  \AttributeTok{N\_of\_test\_user =} \FunctionTok{length}\NormalTok{(}\FunctionTok{unique}\NormalTok{(dsws}\SpecialCharTok{::}\NormalTok{test\_data}\SpecialCharTok{$}\NormalTok{user\_id)),}
  \AttributeTok{N\_of\_test\_level =} \FunctionTok{length}\NormalTok{(}\FunctionTok{unique}\NormalTok{(dsws}\SpecialCharTok{::}\NormalTok{test\_data}\SpecialCharTok{$}\NormalTok{level\_id))}
\NormalTok{)}

\NormalTok{fit }\OtherTok{\textless{}{-}}\NormalTok{ model\_1}\SpecialCharTok{$}\FunctionTok{sample}\NormalTok{(}
  \AttributeTok{data =}\NormalTok{ input\_list,}
  \AttributeTok{iter\_warmup =}\NormalTok{ ITER\_SAMPLING,}
  \AttributeTok{iter\_sampling =}\NormalTok{ ITER\_SAMPLING,}
  \AttributeTok{chains =}\NormalTok{ CHAINS,}
  \AttributeTok{parallel\_chains =}\NormalTok{ CHAINS,}
  \AttributeTok{show\_messages=}\ConstantTok{FALSE}
\NormalTok{)}

\NormalTok{test\_users\_simulated\_try\_model\_1 }\OtherTok{\textless{}{-}} \FunctionTok{get\_simulated\_retry}\NormalTok{(}
\NormalTok{  fit}\SpecialCharTok{$}\FunctionTok{output\_files}\NormalTok{(), }
\NormalTok{  MIN\_TEST\_USER\_ID, }
\NormalTok{  MIN\_TEST\_LEVEL\_ID}
\NormalTok{)}
\end{Highlighting}
\end{Shaded}

\begin{verbatim}
## Rows: 600 Columns: 2759
## -- Column specification --------------------------------------------------------
## Delimiter: ","
## dbl (2759): lp__, accept_stat__, stepsize__, treedepth__, n_leapfrog__, dive...
## 
## i Use `spec()` to retrieve the full column specification for this data.
## i Specify the column types or set `show_col_types = FALSE` to quiet this message.
\end{verbatim}

Lets check the first odd in the table, the odd is 2.46. Lets check is it
logical to play this odd or if not.

\begin{verbatim}
## `summarise()` has grouped output by 'user_id'. You can override using the
## `.groups` argument.
\end{verbatim}

\includegraphics{workshop_notebook_files/figure-latex/unnamed-chunk-10-1.pdf}

\begin{verbatim}
## TableGrob (1 x 2) "arrange": 2 grobs
##   z     cells    name           grob
## 1 1 (1-1,1-1) arrange gtable[layout]
## 2 2 (1-1,2-2) arrange gtable[layout]
\end{verbatim}

\hypertarget{modelling-with-superundo-and-shuffle}{%
\subsubsection{4.2 Modelling with superundo and
shuffle}\label{modelling-with-superundo-and-shuffle}}

New assumption is to retry count in level i by user j, \(t_{ij}\) is
distributed with a geometric distribution
\[  t_{ij} \sim Geometric (p_{ij}) \] where p is the probability of the
successful passing in one retry of level j by user i with shuffle usage
\(shuffle_{ij}\) and superundo usage \(superundo_{ij}\). If we manage to
estimate \(p_{ij}\) we can estimate how many retries will be required by
the user i for level j to pass.

Let's assume \(p_{ij}\) dependent on the level and the user and the
perks used during the level. In other words, not only the levels but the
perks like superundo and shuffle are effective on \(p_{ij}\).

\[  p_{ij} = inv\_logit(intercept + u_i + l_j + b_{superundo} * superundo_{ij}+ b_{shuffle} *shuffle_{ij}) \]

\[  u_i \sim normal (0, alpha_u) \] \[  l_j \sim normal (0, alpha_l) \]
\[  alpha_u \sim exponential (1) \] \[  alpha_l \sim exponential (1) \]
\[  b_{superundo} \sim normal (0, 1) \]
\[  b_{shuffle} \sim normal (0, 1) \]

\(alpha_u\) and \(alpha_l\) are hyper priors for multi level model. We
only observe \(t_{ij}\)while other parameters have latent effects. Check
\url{https://nicholasrjenkins.science/tutorials/bayesian-inference-with-stan/mm_stan/}
for more on stan and multi level modelling.

\begin{Shaded}
\begin{Highlighting}[]
\NormalTok{model\_2 }\OtherTok{\textless{}{-}} \FunctionTok{cmdstan\_model}\NormalTok{(}\StringTok{\textquotesingle{}geometric\_model\_01.stan\textquotesingle{}}\NormalTok{)}


\NormalTok{input\_list }\OtherTok{\textless{}{-}} \FunctionTok{list}\NormalTok{(}
  \AttributeTok{N =} \FunctionTok{nrow}\NormalTok{(train\_data),}
  \AttributeTok{user\_id =}\NormalTok{ train\_data}\SpecialCharTok{$}\NormalTok{user\_id,}
  \AttributeTok{N\_of\_user\_id =} \FunctionTok{max}\NormalTok{(train\_data}\SpecialCharTok{$}\NormalTok{user\_id),}
  
  \AttributeTok{level\_id =}\NormalTok{ dsws}\SpecialCharTok{::}\NormalTok{train\_data}\SpecialCharTok{$}\NormalTok{level\_id }\SpecialCharTok{{-}}\NormalTok{ FIRST\_LEVEL }\SpecialCharTok{+} \DecValTok{1}\NormalTok{,}
  \AttributeTok{N\_of\_level\_id =} \FunctionTok{max}\NormalTok{(dsws}\SpecialCharTok{::}\NormalTok{train\_data}\SpecialCharTok{$}\NormalTok{level\_id) }\SpecialCharTok{{-}}\NormalTok{ FIRST\_LEVEL }\SpecialCharTok{+} \DecValTok{1}\NormalTok{,}
  
  \AttributeTok{retrial\_count =}\NormalTok{ train\_data}\SpecialCharTok{$}\NormalTok{retrial\_count,}
  
  \AttributeTok{N\_of\_test\_user =} \FunctionTok{length}\NormalTok{(}\FunctionTok{unique}\NormalTok{(test\_data}\SpecialCharTok{$}\NormalTok{user\_id)),}
  \AttributeTok{N\_of\_test\_level =} \FunctionTok{length}\NormalTok{(}\FunctionTok{unique}\NormalTok{(test\_data}\SpecialCharTok{$}\NormalTok{level\_id)),}
  \AttributeTok{superundo =}\NormalTok{ dsws}\SpecialCharTok{::}\NormalTok{superundo,}
  \AttributeTok{shuffle =}\NormalTok{ dsws}\SpecialCharTok{::}\NormalTok{shuffle}
\NormalTok{)}

\NormalTok{fit }\OtherTok{\textless{}{-}}\NormalTok{ model\_2}\SpecialCharTok{$}\FunctionTok{sample}\NormalTok{(}
  \AttributeTok{data =}\NormalTok{ input\_list,}
  \AttributeTok{iter\_warmup =}\NormalTok{ ITER\_SAMPLING,}
  \AttributeTok{iter\_sampling =}\NormalTok{ ITER\_SAMPLING,}
  \AttributeTok{chains =}\NormalTok{ CHAINS,}
  \AttributeTok{parallel\_chains =}\NormalTok{ CHAINS,}
  \AttributeTok{show\_messages=}\ConstantTok{FALSE}
\NormalTok{)}

\NormalTok{test\_users\_simulated\_try\_model\_2 }\OtherTok{\textless{}{-}}\NormalTok{ dsws}\SpecialCharTok{::}\FunctionTok{get\_simulated\_retry}\NormalTok{(}
\NormalTok{  fit}\SpecialCharTok{$}\FunctionTok{output\_files}\NormalTok{(), }
\NormalTok{  MIN\_TEST\_USER\_ID, }
\NormalTok{  MIN\_TEST\_LEVEL\_ID}
\NormalTok{)}
\end{Highlighting}
\end{Shaded}

\begin{verbatim}
## Rows: 600 Columns: 2761
## -- Column specification --------------------------------------------------------
## Delimiter: ","
## dbl (2761): lp__, accept_stat__, stepsize__, treedepth__, n_leapfrog__, dive...
## 
## i Use `spec()` to retrieve the full column specification for this data.
## i Specify the column types or set `show_col_types = FALSE` to quiet this message.
\end{verbatim}

\begin{Shaded}
\begin{Highlighting}[]
\NormalTok{model\_2\_play\_vector }\OtherTok{\textless{}{-}}\NormalTok{ test\_users\_simulated\_try\_model\_2 }\SpecialCharTok{\%\textgreater{}\%}
  \FunctionTok{merge}\NormalTok{(dsws}\SpecialCharTok{::}\NormalTok{odds\_table) }\SpecialCharTok{\%\textgreater{}\%} 
  \FunctionTok{group\_by}\NormalTok{(user\_id, level\_id) }\SpecialCharTok{\%\textgreater{}\%}
  \FunctionTok{summarise}\NormalTok{(}\AttributeTok{p=}\FunctionTok{sum}\NormalTok{(value }\SpecialCharTok{\textless{}}\NormalTok{ line)}\SpecialCharTok{/}\FunctionTok{n}\NormalTok{(), }\AttributeTok{odd=}\FunctionTok{median}\NormalTok{(odd)) }\SpecialCharTok{\%\textgreater{}\%}
  \FunctionTok{mutate}\NormalTok{(}\AttributeTok{suggested\_odd=}\NormalTok{(}\DecValTok{1} \SpecialCharTok{{-}}\NormalTok{ p) }\SpecialCharTok{/}\NormalTok{ p }\SpecialCharTok{+} \DecValTok{1}\NormalTok{) }\SpecialCharTok{\%\textgreater{}\%} 
  \FunctionTok{mutate}\NormalTok{(}\AttributeTok{play=}\NormalTok{odd}\SpecialCharTok{\textgreater{}}\NormalTok{suggested\_odd) }\SpecialCharTok{\%\textgreater{}\%} 
  \FunctionTok{pull}\NormalTok{(play)}
\end{Highlighting}
\end{Shaded}

\begin{verbatim}
## `summarise()` has grouped output by 'user_id'. You can override using the
## `.groups` argument.
\end{verbatim}

\begin{Shaded}
\begin{Highlighting}[]
\NormalTok{dsws}\SpecialCharTok{::}\FunctionTok{score\_the\_play\_vector}\NormalTok{(model\_2\_play\_vector, }\AttributeTok{scores =}\NormalTok{ dsws}\SpecialCharTok{::}\NormalTok{scores, }\AttributeTok{odds\_table =}\NormalTok{ dsws}\SpecialCharTok{::}\NormalTok{odds\_table)}
\end{Highlighting}
\end{Shaded}

\includegraphics{workshop_notebook_files/figure-latex/unnamed-chunk-12-1.pdf}

\begin{verbatim}
## TableGrob (1 x 2) "arrange": 2 grobs
##   z     cells    name           grob
## 1 1 (1-1,1-1) arrange gtable[layout]
## 2 2 (1-1,2-2) arrange gtable[layout]
\end{verbatim}

\hypertarget{real-difficulty}{%
\subsection{5. Real difficulty}\label{real-difficulty}}

In section 3 we try to check if the game becomes harder or not by
checking the retry counts per levels. Yet we could not see that it is
the case. The blue line that shows the trend has very low derivative.
Now we have difficulties as model results. Lets check it is still in the
case or not?

\begin{Shaded}
\begin{Highlighting}[]
\NormalTok{levels }\OtherTok{\textless{}{-}} \FunctionTok{vroom}\NormalTok{(fit}\SpecialCharTok{$}\FunctionTok{output\_files}\NormalTok{(), }\AttributeTok{comment =} \StringTok{\textquotesingle{}\#\textquotesingle{}}\NormalTok{, }\AttributeTok{delim =} \StringTok{\textquotesingle{},\textquotesingle{}}\NormalTok{) }\SpecialCharTok{\%\textgreater{}\%}
  \FunctionTok{select}\NormalTok{(}\FunctionTok{starts\_with}\NormalTok{(}\StringTok{\textquotesingle{}l.\textquotesingle{}}\NormalTok{))  }\SpecialCharTok{\%\textgreater{}\%}
\NormalTok{  tidyr}\SpecialCharTok{::}\FunctionTok{pivot\_longer}\NormalTok{(}\FunctionTok{everything}\NormalTok{()) }\SpecialCharTok{\%\textgreater{}\%}
\NormalTok{  tidyr}\SpecialCharTok{::}\FunctionTok{separate}\NormalTok{(name, }\FunctionTok{c}\NormalTok{(}\StringTok{"\_"}\NormalTok{, }\StringTok{"level\_id"}\NormalTok{), }\AttributeTok{sep =} \StringTok{"}\SpecialCharTok{\textbackslash{}\textbackslash{}}\StringTok{."}\NormalTok{) }\SpecialCharTok{\%\textgreater{}\%}
  \FunctionTok{mutate}\NormalTok{(}\AttributeTok{level\_id =} \FunctionTok{as.numeric}\NormalTok{(level\_id) }\SpecialCharTok{+} \DecValTok{100}\NormalTok{) }\SpecialCharTok{\%\textgreater{}\%} 
  \FunctionTok{group\_by}\NormalTok{(level\_id)}\SpecialCharTok{\%\textgreater{}\%}
  \FunctionTok{summarise}\NormalTok{(}\AttributeTok{m=}\FunctionTok{mean}\NormalTok{(value))}
\end{Highlighting}
\end{Shaded}

\begin{verbatim}
## Rows: 600 Columns: 2761
## -- Column specification --------------------------------------------------------
## Delimiter: ","
## dbl (2761): lp__, accept_stat__, stepsize__, treedepth__, n_leapfrog__, dive...
## 
## i Use `spec()` to retrieve the full column specification for this data.
## i Specify the column types or set `show_col_types = FALSE` to quiet this message.
\end{verbatim}

\begin{Shaded}
\begin{Highlighting}[]
\FunctionTok{ggplot}\NormalTok{(}\AttributeTok{data=}\NormalTok{levels, }\FunctionTok{aes}\NormalTok{(}\AttributeTok{x=}\NormalTok{level\_id, }\AttributeTok{y=}\NormalTok{ m)) }\SpecialCharTok{+} 
  \FunctionTok{geom\_line}\NormalTok{() }\SpecialCharTok{+}
  \FunctionTok{geom\_smooth}\NormalTok{(}\AttributeTok{span =} \FloatTok{0.1}\NormalTok{)}\SpecialCharTok{+}
  \FunctionTok{geom\_smooth}\NormalTok{(}\AttributeTok{method =}\NormalTok{ lm, }\AttributeTok{se =} \ConstantTok{FALSE}\NormalTok{)}
\end{Highlighting}
\end{Shaded}

\begin{verbatim}
## `geom_smooth()` using method = 'loess' and formula 'y ~ x'
## `geom_smooth()` using formula 'y ~ x'
\end{verbatim}

\includegraphics{workshop_notebook_files/figure-latex/unnamed-chunk-13-1.pdf}

There is now a new clear trend, do you have any idea why?

\end{document}
